\documentclass[12pt,a4paper]{report}
\usepackage{listings}
\usepackage{hyperref}
\usepackage{graphicx}
\usepackage{float}
\usepackage{url}
\usepackage[dvipsnames]{xcolor}
\usepackage[table]{xcolor}
\usepackage{amsmath}
\usepackage{breqn}
\usepackage{adjustbox}
\usepackage{makecell}
\usepackage[a4paper, margin=2cm]{geometry}
\definecolor{DarkPurpleMamado}{RGB}{128,0,128}
\definecolor{PurpleNoMamado}{RGB}{128,70,128}
\lstset{
    language=C,
    basicstyle=\footnotesize,
    numbers=left,
    stepnumber=1,
    showstringspaces=false,
    tabsize=1,
    breaklines=true,
    breakatwhitespace=false,
}

\def \unidad{Escuela de Ingeniería en Computación}
\def \programa{Ingeniería en Computación}
\def \curso{IC-6400 - Investigación de Operaciones}
\def \titulo{Project 04}
\def \subtitulo{Simplex Report}
\def \autores{
    Gerald Calderón Castro\\
    gecalderon@estudiantec.cr\\
    2023125197\\

    \vspace{0.5cm}

    Óscar Obando Umaña\\
    osobando@estudiantec.cr\\
    2023091684
}
\def \fecha{November 2025}
\def \lugar{San José , Costa Rica}

\begin{document}

\begin{titlepage}
    \begin{center}
        \vspace*{1cm}

        % La imagen está en el mismo directorio que el .tex
        \includegraphics[width=0.8\linewidth]{logo_tec.jpg}\\

        \LARGE
        \unidad\\
        \programa\\
        \curso

        \vspace{1cm}

        \Huge
        \textbf{\titulo}

        \vspace{0.5cm}
        \LARGE
        \subtitulo

        \vspace{1.5cm}

        \large
        \autores

        \vfill

        \lugar\\
        \fecha

    \end{center}
\end{titlepage}
\tableofcontents

\section{Description}
The simplex algorithm is a very simple algorithm.
\section{Solving Problem Name}
\subsection{Mathematical representation}
\textbf{Minimize}
\begin{dmath}
z = 0,00000q+0,00000e+0,00000x_{3q+0,00000e+0,00000t\end{dmath}
\section{The initial simplex table}
\end{document}