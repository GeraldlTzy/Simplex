\documentclass[12pt,a4paper]{report}
\usepackage{listings}
\usepackage{hyperref}
\usepackage{graphicx}
\usepackage{float}
\usepackage{url}
\usepackage[dvipsnames]{xcolor}
\usepackage[table]{xcolor}
\usepackage{amsmath}
\usepackage{breqn}
\usepackage{adjustbox}
\usepackage{makecell}
\usepackage[a4paper, margin=2cm]{geometry}
\definecolor{DarkPurpleMamado}{RGB}{128,0,128}
\definecolor{PurpleNoMamado}{RGB}{128,70,128}
\lstset{
    language=C,
    basicstyle=\footnotesize,
    numbers=left,
    stepnumber=1,
    showstringspaces=false,
    tabsize=1,
    breaklines=true,
    breakatwhitespace=false,
}

\def \unidad{Escuela de Ingeniería en Computación}
\def \programa{Ingeniería en Computación}
\def \curso{IC-6400 - Investigación de Operaciones}
\def \titulo{Project 04}
\def \subtitulo{Simplex Report}
\def \autores{
    Gerald Calderón Castro\\
    gecalderon@estudiantec.cr\\
    2023125197\\

    \vspace{0.5cm}

    Óscar Obando Umaña\\
    osobando@estudiantec.cr\\
    2023091684
}
\def \fecha{November 2025}
\def \lugar{San José , Costa Rica}

\begin{document}

\begin{titlepage}
    \begin{center}
        \vspace*{1cm}

        % La imagen está en el mismo directorio que el .tex
        \includegraphics[width=0.8\linewidth]{logo_tec.jpg}\\

        \LARGE
        \unidad\\
        \programa\\
        \curso

        \vspace{1cm}

        \Huge
        \textbf{\titulo}

        \vspace{0.5cm}
        \LARGE
        \subtitulo

        \vspace{1.5cm}

        \large
        \autores

        \vfill

        \lugar\\
        \fecha

    \end{center}
\end{titlepage}
\tableofcontents

\chapter{Simplex Description}
The simplex algorithm is a very simple algorithm.
\chapter{Solving mofongo}
\section{Mathematical representation}
\textbf{Minimize:}
\begin{dmath}
z = 2.00000Feliz + -3.00000Jueves 
\end{dmath}
\textbf{Subject To:}
\begin{dmath}
1.00000Feliz + 1.00000Jueves \leq4.00000
\end{dmath}
\begin{dmath}
1.00000Feliz + -1.00000Jueves \leq6.00000
\end{dmath}
\section{The initial simplex table}
\begin{center}
\begin{adjustbox}{max width=1\textwidth,keepaspectratio}
\begin{tabular}{|c|c|c|c|c|c|} 
 \hline 
$Z$ & $Feliz$ & $Jueves$ & $s_{1}$ & $s_{2}$ & $B$\\ 
 \hline 
1.00000 & -2.00000 & 3.00000 & 0.00000 & 0.00000 & 0.00000\\ 
 \hline 
0.00000 & 1.00000 & 1.00000 & 1.00000 & 0.00000 & 4.00000\\ 
 \hline 
0.00000 & 1.00000 & -1.00000 & 0.00000 & 1.00000 & 6.00000\\ 
 \hline 
\end{tabular} 
\end{adjustbox}
\end{center}
Pivoteo(1)
\begin{center}
\begin{adjustbox}{max width=1\textwidth,keepaspectratio}
\begin{tabular}{|c|c|c|c|c|c|} 
 \hline 
$Z$ & $Feliz$ & $Jueves$ & $s_{1}$ & $s_{2}$ & $B$\\ 
 \hline 
1.00000 & -5.00000 & 0.00000 & -3.00000 & 0.00000 & -12.00000\\ 
 \hline 
0.00000 & 1.00000 & 1.00000 & 1.00000 & 0.00000 & 4.00000\\ 
 \hline 
0.00000 & 2.00000 & 0.00000 & 1.00000 & 1.00000 & 10.00000\\ 
 \hline 
\end{tabular} 
\end{adjustbox}
\end{center}
\section{The initial simplex table}
\end{document}